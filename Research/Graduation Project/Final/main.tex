\documentclass[
12pt,
openany,
twoside,
a4paper,
englishwr,
english
]{ppgca}

%\usepackage[style=ieee,backend=biber,citestyle=numeric]{biblatex}
\usepackage{csquotes}
\usepackage{enumitem}
\usepackage{float}
\usepackage{tcolorbox}
\usepackage[left=2.5cm,right=2.5cm,top=3cm,bottom=2cm]{geometry}

%\addbibresource{refs.bib}

% User's custom commands
\setcounter{secnumdepth}{4}
\setcounter{tocdepth}{4}

% Info
\title{Comparing the LMC Complexity of Neural Networks with their Generalization Capability}
\author{Lucas Miranda Mendonça Rezende}
\local{Ribeirão Preto--SP}
\data{2025}
\orientador{Ph.D. Luiz Otavio Murta Junior}
\tipotrabalho{Dissertation}

\preambulo{Monograph presented to the Faculty of Philosophy, Sciences and Letters of Ribeirão Preto (FFCLRP) from the University of São Paulo (USP), as part of the requirements to hold the Bachelor of Science degree.}

\makeindex

\begin{document}

\selectlanguage{english}
\renewcommand{\contentsname}{Table of Contents}
\renewcommand{\bibname}{References}

\imprimircapa
\imprimirfolhaderosto

% Abstract
\begin{resumo}[Abstract]
    Current scaling laws suggest that model performance improves with increased parameters, dataset size, and compute, but these improvements follow power laws requiring exponential resources for constant gains. Understanding the learning process through alternative metrics could enable architectural improvements and optimization during training. This work investigates whether LMC statistical complexity, a measure combining Shannon entropy and disequilibrium of weight distributions, correlates with neural network generalization capability. We analyzed 35 open-weight transformer-based language models from Meta, Google, Microsoft, and OpenAI, ranging from 117 million to 150 billion parameters, computing LMC complexity across different weight-type combinations (bias, norm, embedding, other) and filtering settings. Model performance was evaluated using four benchmarks: MMLU, MMLU-Pro, OpenLLM, and LMArena. Results revealed that, despite statistical significance in aggregated analyses and specific benchmarks like MMLU, a general correlation between LMC complexity and benchmark performance cannot be confirmed. The observed positive trends were often driven by outliers or showed inconsistent patterns, suggesting that LMC complexity might not be a reliable predictor of generalization capability in the studied context.
\end{resumo}

\pdfbookmark[0]{\contentsname}{toc}
\tableofcontents*
\cleardoublepage

\textual

\section{Introduction}

Since the creation of Transformers in 2017 \cite{Vaswani2017} and the subsequent usage of this new technique in the training of Large Language Models (LLMs) , there has been a gold rush within the machine learning world. GPT-3.5, the original ChatGPT model, rapidly gained widespread adoption becoming the fastest-growing consumer application in history after its launch in 2022 \cite{ChatGPTAdoption2022} sparking further interest in researchers, investors and the general public for more powerful and cost-efficient models. 

Since then multiple new models have been developed by the biggest technology companies in the world such as Google, Microsoft, NVidia, Amazon and amazingly also by some smaller ones such as DeepSeek. As better models in almost every metric emerged, it also became increasingly obvious that all this improvement wasn’t for free: billions were spent in larger datacenters, predictions that we soon wouldn’t have enough data on the internet to keep building bigger models, increasingly expensive for marginal performance gains. But why?

In 2020, before the launch of the original ChatGPT, Researchers at openAI \cite{Kaplan2020} have found that "performance improves smoothly as we increase the model size N (the number of parameters excluding embeddings), dataset size D, and amount of compute C" and that "performance depends strongly on scale and weakly on model shape, such as depth vs. width". Secondly, it was also observed that different architectures could impact training performance, e.g. Transformers would have lower test loss than LSTMs. Those statements become the basis for the scaling "laws" we currently accept.

It is evident that the most straightforward way to get a better model, at least considering performance in terms of lower test loss, is to increase the scale (N, C and D sizes), however, there’s a huge drawback: the scaling laws described in this case are power laws, meaning we would need an exponential amount of resources to achieve a constant gain in performance. The second approach would be creating better architectures or improving the existing ones, which demands research and it’s generally not as simple as increasing a number.

The first approach, being the easiest, was explored by the companies creating the new models, pushing those three variables to ludicrous amounts. The GPT family, mentioned above, serves as a good example: the original GPT had 117 million parameters, GPT-2 had 1.5 billion (a 12x increase from GPT) and GPT-3 had 175 billion (a 117x increase from GPT-2) \cite{Brown2020} \cite{Radford2019} \cite{Radford2018}. It is noticeable that, just a few years later, the industry is already showing signs of exhaustion: new models do not exhibit performance improvements as dramatic as those observed in the recent past, although investment in training infrastructure has never been higher.

The second approach is what we intend to contribute in this work: an improvement in the architecture itself. Of course, before creating a new revolutionary architecture it would be useful to understand how machines learn. In a typical analysis setting, knowing how the process works is a requirement to engineer a better version of it. In Machine Learning, however, the process of learning, which is somehow connected to the well-known process of training, is still far from being fully understood.

Discussing the understanding of the learning process is out of the scope of this work, yet we are going to analyze what may be a piece of the puzzle: the LMC statistical complexity \cite{LopezRuiz1995}, a metric that appears to be related to the model's performance and might help understanding the behaviour of such by proving insights about the weights distribution.

\subsection{Work thesis}

    The main hypothesis of this work comes from Professor Luiz Otavio Murta Junior \cite{MurtaJunior2025} and can be summarized as: \textbf{"There exists a relationship between model complexity and its inference capability"}.

\subsection{Objective}

    Validating the work thesis means we would have a new way of indirectly assessing a model's performance by just looking at the distribution of its weights, opening the gates for new optimization processes during training, potentially reducing the amount of compute necessary to reach the best performance or even finding new maximums. Also, having a validated weights distribution x performance comparison should improve the data richness when studying a model's learning process in future studies. Thus, we can escalate the following objectives for this work:
    \begin{itemize}
        \item Validate the existence of a meaningful relationship between complexity of neural network weights and their inference performance.
        \item If possible, find the mathematical relation between those measures.
        \item Explore other dimensions of the problem that can affect complexity and performance measures such as parameter count.
    \end{itemize}

\section{Methodology}

\subsection{Testing Environment}
\label{sec:experimental_setup}

We used the following computational environment for our experiments:
\begin{itemize}
    \item OS: Linux 6.8.0-64-generic
    \item CPU: 2x Intel Xeon Gold 6130 @ 2.1GHz - 32 cores / 64 threads per CPU, 22M cache each
    \item RAM: 512GB DIMM 2400 MHz (DDR4)
    \item GPU: NVIDIA Quadro P5000 (GP104GL) - 16GB VRAM
    \item Storage: RAID5 (2x4TB) 8TB SAS with Broadcom MR9260-8i controller + 512GB RAID0 SSD
\end{itemize}
The hardware was provided for research purposes by Universidade de São Paulo. The choice of this environment was motivated by the availability of a high-performance GPU and large quantities of RAM necessary for handling large language models.

We used the following tools and libraries:
\begin{itemize}
    \item Python 3.12.3
    \item PyTorch 2.8.0 \cite{Paszke2019}
    \item Transformers 4.56.2 \cite{Wolf2019}
    \item Hugging Face Hub 0.35.0 \cite{HuggingFaceHub}
    \item Numpy 2.2.6
    \item Pandas 2.3.2
    \item Pysr 1.5.9
    \item SciPy 1.16.1 \cite{SciPy2020}
    \item Matplotlib 3.10.6
    \item Seaborn 0.13.2 \cite{Waskom2021}
    \item CUDA Toolkit 12.6
\end{itemize}
The choice of versions was motivated by compatibility with our computational environment's GPU.

\subsection{Model Selection}

Due to the wide availability of open models combined with the ease of accessing them through the Transformers library \cite{Wolf2019} and Hub library \cite{HuggingFaceHub}, \textbf{Hugging Face} was chosen as the source for model selection in this study.

Hugging Face is a platform that hosts a variety of machine learning models, datasets, and tools. It is widely used in the AI research community for sharing and collaborating on machine learning projects \cite{HuggingFaceResearchGate, HuggingFaceMain}. Other platforms such as Ollama \cite{Ollama} were considered, but ultimately not chosen due to the limited number of models available and fewer contributing companies. 

Model selection proceeded in two stages:

\begin{enumerate}
    \item First, we identified major technology companies by market capitalization that publish openly released language models. The companies considered were \textbf{OpenAI}, \textbf{Google}, \textbf{Meta}, and \textbf{Microsoft}.
    \item Second, for each company we compiled a candidate set consisting of every model that satisfied the following criteria:
    
        \begin{itemize}
            \item Must be available on the Hugging Face platform on the official company account.
            \item The model is a transformer-based language model.
            \item Model weights are publicly accessible (open weights), including models released behind
            gated access.
            \item The model is text-only (no multimodal image/audio inputs).
            \item The model is an original base model rather than a task-specific fine-tuned variant.
            \item The total parameter count is below 150 billion. This upper bound was imposed due to
            hardware and inference limitations in the computational environment used for our experiments.
            \item The model is supported by the Hugging Face Transformers library \cite{Wolf2019} (i.e., it can be
            instantiated via the AutoModel utility), which ensures consistent loading and
            preprocessing across the candidate set.
            \item There is at least one publicly available benchmark result for the model among the selected benchmarks (section \ref{sec:benchmark_selection}).
        \end{itemize}

\end{enumerate}

The final selection of models used in this study is listed in Table \ref{tab:selected_models}.

\begin{table}[H]
    \begin{tabular}{p{0.24\linewidth} p{0.24\linewidth} p{0.24\linewidth} p{0.24\linewidth}}
        \hline
        Meta & Google & Microsoft & OpenAI \\
        \hline
        meta-llama/Llama-4-Scout-17B-16E & google/gemma-3-27b-pt & microsoft/Phi-4-mini-reasoning & openai/gpt-oss-120b \\
        meta-llama/Llama-3.2-3B & google/gemma-3-12b-pt & microsoft/Phi-4-reasoning & openai/gpt-oss-20b \\
        meta-llama/Llama-3.2-1B & google/gemma-3-4b-pt & microsoft/Phi-4-reasoning-plus & openai-community/gpt2-xl \\
        meta-llama/Llama-3.1-70B & google/gemma-3-1b-pt & microsoft/phi-4 & openai-community/gpt2-large \\
        meta-llama/Llama-3.1-8B & google/gemma-3-270m & microsoft/phi-2 & openai-community/gpt2-medium \\
        meta-llama/Meta-Llama-3-70B & google/gemma-2-27b & microsoft/phi-1\_5 & openai-community/gpt2 \\
        meta-llama/Meta-Llama-3-8B & google/gemma-2-9b & microsoft/phi-1 & \\
        meta-llama/Llama-2-70b-hf & google/gemma-2-2b &  & \\
        meta-llama/Llama-2-13b-hf & google/gemma-7b &  & \\
        meta-llama/Llama-2-7b-hf & google/gemma-2b &  & \\
            & google/recurrentgemma-9b &  & \\
            & google/recurrentgemma-2b &  & \\
        \hline
    \end{tabular}
    \caption{Selected language models included in this study.}
    \label{tab:selected_models}
\end{table}

\begin{tcolorbox}[colback=gray!10, colframe=black, boxrule=1pt, rounded corners]
    \textbf{Meta Models}:
    \cite{MetaLlama4Scout}
    \cite{MetaLlama3.2-3B}
    \cite{MetaLlama3.2-1B}
    \cite{MetaLlama3.1-70B}
    \cite{MetaLlama3.1-8B}
    \cite{MetaLlama3-70B}
    \cite{MetaLlama3-8B}
    \cite{MetaLlama2-70b}
    \cite{MetaLlama2-13b}
    \cite{MetaLlama2-7b}

    \textbf{Google Models}:
    \cite{GoogleGemma3-27b}
    \cite{GoogleGemma3-12b}
    \cite{GoogleGemma3-4b}
    \cite{GoogleGemma3-1b}
    \cite{GoogleGemma2-27b}
    \cite{GoogleGemma2-9b}
    \cite{GoogleGemma2-2b}
    \cite{GoogleGemma-7b}
    \cite{GoogleGemma-2b}
    \cite{GoogleRecurrentGemma-9b}
    \cite{GoogleRecurrentGemma-2b}

    \textbf{Microsoft Models}:
    \cite{MicrosoftPhi4-mini-reasoning}
    \cite{MicrosoftPhi4-reasoning}
    \cite{MicrosoftPhi4-reasoning-plus}
    \cite{MicrosoftPhi4}
    \cite{MicrosoftPhi2}
    \cite{MicrosoftPhi1-5}
    \cite{MicrosoftPhi1}

    \textbf{OpenAI Models}:
    \cite{OpenAIGPT-oss-120b}
    \cite{OpenAIGPT-oss-20b}
    \cite{OpenAIGPT2-xl}
    \cite{OpenAIGPT2-large}
    \cite{OpenAIGPT2-medium}
    \cite{OpenAIGPT2}
\end{tcolorbox}

\subsection{LMC Complexity}
\label{sec:lmc_complexity}

    According to \cite{LopezRuiz1995}, LMC Statistical Complexity is the product of two other measures: Disequilibrium and Shannon entropy. It captures both the structured and unstructured aspects of the distribution:
    \[ C_{LMC} = H \times D \]

    Disequilibrium measures how far a probability distribution is from being uniform, quantifying the "order" or structure in the data. It is calculated as:
    \[ D = \sum_{i=1}^{n} \left(p_i - \frac{1}{n}\right)^2 \]

    Shannon entropy measures the amount of uncertainty or randomness in a probability distribution. The normalized Shannon entropy is given by:
    \[ H = -K \sum_{i=1}^{n} p_i \log p_i \]

    The values \( p_i \) represent the probabilities associated with each state \( i \) in the distribution, and \( n \) is the total number of states. K is a positive constant and, in our case, is set to 1 for simplicity. K can be changed later since \( C_{LMC} = (-K \sum_{i=1}^{n} p_i \log p_i)  \times D \) is equivalent to \( C_{LMC} = K \times (- \sum_{i=1}^{n} p_i \log p_i)  \times D \).

    \subsubsection{Reading Model Weights}
    \label{sec:reading_model_weights}

        Model weights are fetched using the Hugging Face Hub library \cite{HuggingFaceHub} and loaded using the Transformers library \cite{Wolf2019}. For each selected model, we instantiate the model using the \textbf{AutoModel} utility, which automatically handles model architecture loading and weight initialization. 

        The models are loaded entirely into the main memory instead of a GPU since the amount of VRAM available in the computational environment is insufficient for larger models $(> 70 \text{ billion parameters})$. Other problems such as handling symbolic tensors also motivated this decision.

        Using the main memory is considerably slower than using a GPU, but allows us to work with larger models without running into memory limitations. Since we had a large amount of RAM, we decided to cast all the model weights to float32 during the AutoModel instantiation as CPUs generally handle this format natively and, as a consequence, make calculations faster.
        
        Once loaded, we extract all the parameters from the model using the \texttt{named\_parameter()} method provided by the Transformers library. This method returns an iterator over all model parameters, each parameter being represented as a tensor. 

        We then flatten each tensor into a one-dimensional array and store it in a list. Each array is labeled as one of the following \textbf{weight-type categories}:
        \begin{itemize}
            \item Bias: if 'bias' is in the parameter name
            \item Norm: if 'norm' is in the parameter name
            \item Embedding: if 'embed' is in the parameter name
            \item Other: all other weights
        \end{itemize}

        These categories are the most common types of weights found in this architecture and will be used later as another dimension of analysis. 
        
        The choice of also analyzing by weight type is motivated by the fact that different weight types may exhibit different statistical properties and, as a consequence, different complexity characteristics. Studies such as the already cited OpenAI's \cite{Kaplan2020} take into account different weight types when analyzing scaling laws.

    \subsubsection{Filtering}
    \label{sec:filtering}

        Casting different encoding formats to float32 may introduce rounding errors that manifest as extreme outliers in the weight distribution. This is rare (approximately 10 to 30 values in 100 billion) but can be problematic for our next step: histogram construction (section \ref{sec:data_discretization}), since the number of bins will be affected by the range of values in the data. The reasons for these rounding errors could not be fully investigated within the scope of this work, but they are likely related to floating point rounding errors.

        To mitigate the impact of outliers, we apply a value removal approach. This process filters the data to retain only values within a configurable range centered around the mean:

        \[ \text{lower bound} = \mu - \sigma_{\text{filter}} \cdot \sigma \]
        \[ \text{upper bound} = \mu + \sigma_{\text{filter}} \cdot \sigma \]

        where \( \mu \) is the data mean, \( \sigma \) is the data standard deviation, and \( \sigma_{\text{filter}} \) is a configurable parameter that controls the filtering strength. Values falling outside this range are excluded from further analysis. 

        It is not trivial to choose the best value for \( \sigma_{\text{filter}} \). A very low value may remove important parts of the distribution, while a very high value may not effectively mitigate the outlier problem. For this reason, we experiment with different values of \( \sigma_{\text{filter}} \) and analyze how they affect the final complexity results. As a consequence, this becomes yet another dimension of analysis in our study.

        The values of \( \sigma_{\text{filter}} \) tested will be: \textbf{0.125}, \textbf{0.25}, \textbf{0.5}, \textbf{1}, \textbf{2}, \textbf{3}, \textbf{4}, \textbf{5}, \textbf{10}, \textbf{20}. The choice of \( \sigma_{\text{filter}} \) that will be used as the filtering option for the next steps is the one that: 
        \begin{enumerate}
            \item most reduces the bin count compared to the unfiltered data; 
            \item has the least amount of filtering or, in other words, the highest value of \( \sigma_{\text{filter}} \) among the ones chosen in the first criterion.
        \end{enumerate}

    \subsubsection{Data Discretization and Histogram}
    \label{sec:data_discretization}

        To compute the LMC complexity of a finite array of floating-point numbers, we first construct a histogram to discretize the data into a probability distribution. This is justified as the chance of finding two identical numbers is extremely low and, as a consequence, it is hard to determine the probability of each value. 
        
        We will call the set of all data points as \( S \), the total amount of data points as \( N \) and the number of bins they will be distributed into as \( n \). The probabilities \( p_i \) are then calculated as \( p_i = \frac{f_i}{N} \) where \( f_i \) is the frequency count of data points in bin \( i \).

        As expected, this approach revisits a classic issue in histogram-based analysis: the choice of the number of bins \( n \) will impact the resulting probability distribution, which is particularly problematic for the LMC complexity measure; variations in \( n \) can cause significant fluctuations in the final result. Selecting an inappropriate number of bins may produce misleading values, either by oversimplifying the distribution (too few bins) or by introducing noise (too many bins). 

        There are a variety of methods to determine \( n \) in a histogram, most of them with their own advantages and disadvantages. Commonly used methods such as \textbf{Sturges' formula} \cite{Sturges1926} and \textbf{Rice Rule} \cite{RiceRule} rely only on the number of data points, while others like \textbf{Scott's normal reference rule} and \textbf{Freedman-Diaconis' choice} also take into account the data distribution by using standard deviation and interquartile range, respectively \cite{Knuth2006}.

        We chose to use the \textbf{Freedman-Diaconis' choice} as it adapts better to our needs. This is justified since \( N \) often consists of billions of numbers, and the distribution, although mostly concentrated between -1 and 1, can become sparse due to outliers and, as a consequence, requires a larger number of bins to capture its characteristics accurately. The Freedman-Diaconis rule helps mitigate the influence of outliers by using the interquartile range \( IQR \) to determine bin width \( h \). The rule is defined as follows \cite{FreedmanDiaconis1981}:

        \[ h = \frac{2 \times IQR}{N^{1/3}} \]

        where IQR is the interquartile range of the data which is calculated as \( Q3 - Q1 \), \( Q3 \) and \( Q1 \) are the values at the 75th and 25th percentiles in the data, respectively. Since \( N \) is very large, the percentiles are computed based on a random sample of 100000 data points to reduce computational cost. The sample size was determined empirically to be large enough to provide stable estimates of the percentiles; the final number of bins showed no variance between tests.

        The number of bins \( n \) can then be calculated as:

        \[ n = \frac{\max(S) - \min(S)}{h} \]

        Finally, we can use the already filtered data and the calculated \( n \) to build a histogram using a simple function provided by PyTorch \cite{Paszke2019}. Then, compute the probabilities \( p_i \) for each bin \( i \).

    \subsubsection{Complexity Calculation}

        With the probabilities \( p_i \) computed from the histogram, we can now calculate the LMC complexity \( C_{LMC} \) using the formulas provided in the beginning of section \ref{sec:lmc_complexity}. This involves calculating the Disequilibrium \( D \) and the Shannon entropy \( H \) using the probabilities, and then multiplying them to obtain the final complexity measure \( C_{LMC} \).

\subsection{Inference Capability}

    Often called model performance, inference capability refers to how well a trained neural network performs on unseen data. This is typically measured using various metrics depending on the specific task the model is designed for. 
    
    In the previously cited research paper by OpenAI \cite{Kaplan2020}, performance was associated with test cross-entropy loss. This metric quantifies the difference between the predicted probability distribution output by the model and the true distribution of the target labels. Lower cross-entropy loss values indicate better model performance, as they reflect a closer alignment between predictions and actual outcomes.

        Unfortunately, not all models we intend to analyze have publicly available training and test data. It is also not possible to run models in a training setting due to performance limitations of the computational environment available. Therefore, we have to rely on imperfect proxies for performance such as \textbf{benchmarks}.
    \subsubsection{Benchmark Selection}
    \label{sec:benchmark_selection}

        Benchmarks are standardized tests designed to evaluate the performance of machine learning models across various tasks. They provide a common ground for comparison by measuring how well different models perform on the same datasets using predefined metrics. 
        
        According to \cite{Owen2024}, benchmarks such as the famous MMLU correlate fairly well with the predicted test loss determined by scaling laws, making them suitable proxies.

        They are, however, not perfect. Some benchmarks may fail to capture all aspects of a model's capabilities, leading to an incomplete assessment of performance. Other problems such as data contamination \cite{Magar2022} can influence the validity of results and make it difficult to fairly compare models created at different times. It is also hard to find benchmarks that are widely reported for all models we intend to analyze.

        A nice proposal for future research to avoid the problems cited above would involve training a model from scratch twice: optimizing it initially for minimal loss and then for minimal loss plus maximum LMC complexity, then comparing the results. This is, however, out of the scope of this work.

        The benchmark selection followed three main heuristics:
        \begin{itemize}
            \item \textbf{Relevance}: The benchmark should be widely recognized and accepted in the machine learning community, e.g., used in major research papers.
            \item \textbf{Generality}: It should cover a range of tasks and data types to provide an assessment of model performance across different scenarios, that is, not being specialized in one task.
            \item \textbf{Availability}: The benchmark results should be publicly available. Benchmarks that cover more of the selected models were preferred.
        \end{itemize}

        Based on those heuristics, the benchmarks selected were:
        \begin{itemize}
            \item \textbf{MMLU (5-shot)}: Massive Multitask Language Understanding, a benchmark that tests models across 57 tasks spanning various subjects and difficulty levels. Widely used to evaluate LLMs \cite{Hendrycks2020}.

            \item \textbf{MMLU-Pro (5-shot)}: An enhanced version of MMLU that includes additional tasks and updated datasets to provide a better evaluation \cite{MMPro2024}.
            
            \item \textbf{OpenLLM (Average)}: A benchmark suite that evaluates models on a variety of tasks, including language understanding, generation, and reasoning. It aggregates results from multiple datasets to provide an overall performance score \cite{Myrzakhan2024}.
            
            \item \textbf{LMArena (Score)}: An online platform where users can submit questions and receive responses from two anonymous large language models (LLMs), then vote on which answer they prefer, helping to crowdsource human preferences for evaluating and ranking LLMs. Its evaluation works by collecting thousands of pairwise comparisons from users, using the Bradley-Terry system to estimate win rates and compute rankings/scores \cite{Chiang2024}.
        \end{itemize}

    \subsubsection{Benchmark Collection Procedure}
    \label{sec:benchmark_collection_procedure}

        Benchmark values were manually collected from multiple sources, in the following order of priority:
        \begin{enumerate}
            \item Official \textit{Hugging Face} model pages
            \item Original research papers
            \item Official websites
            \item Third-party websites
        \end{enumerate}

        \begin{tcolorbox}[colback=gray!10, colframe=black, boxrule=1pt, rounded corners]
            \textbf{Meta Models}: 
            \cite{MetaLlama4Scout}, 
            \cite{MetaLlama3.2-3B}, 
            \cite{MetaLlama3.2-1B}, 
            \cite{MetaLlama3.1-70B}, 
            \cite{MetaLlama3.1-8B}, 
            \cite{MetaLlama3-70B}, 
            \cite{MetaLlama3-8B}, 
            \cite{MetaLlama2-70b}, 
            \cite{MetaLlama2-13b}, 
            \cite{MetaLlama2-7b}
            
            \textbf{Google Models}: 
            \cite{GoogleGemma3-27b}, 
            \cite{GoogleGemma3-12b}, 
            \cite{GoogleGemma3-4b}, 
            \cite{GoogleGemma3-1b}, 
            \cite{GoogleGemma2-27b}, 
            \cite{GoogleGemma2-9b}, 
            \cite{GoogleGemma2-2b}, 
            \cite{GoogleGemma-7b}, 
            \cite{GoogleGemma-2b}, 
            \cite{GoogleRecurrentGemma-9b}, 
            \cite{GoogleRecurrentGemma-2b}
            
            \textbf{Microsoft Models}: 
            \cite{MicrosoftPhi4-mini-reasoning}, 
            \cite{MicrosoftPhi4-reasoning}, 
            \cite{MicrosoftPhi4-reasoning-plus}, 
            \cite{MicrosoftPhi4}, 
            \cite{MicrosoftPhi2}, 
            \cite{MicrosoftPhi1-5}, 
            \cite{MicrosoftPhi1}, 
            \cite{Phi1.5}, 
            \cite{Phi2Blog}
            
            \textbf{OpenAI Models}: 
            \cite{OpenAIGPT-oss-120b}, 
            \cite{OpenAIGPT-oss-20b}, 
            \cite{OpenAIGPT2-xl}, 
            \cite{OpenAIGPT2-large}, 
            \cite{OpenAIGPT2-medium}, 
            \cite{OpenAIGPT2}, 
            \cite{OpenAIOpenModels},
            
            \textbf{Additional References}: 
            \cite{HuggingFaceMain}, 
            \cite{Ollama},
            \cite{LLMExplorer}, 
            \cite{RankedAGI},
        \end{tcolorbox} 

        If more than one source was available for the same model and benchmark, the one with higher priority was chosen. Not all models had results available for all benchmarks. Figure \ref{fig:benchmark_availability} shows the availability matrix.

        \begin{figure}[H]
            \centering
            \includegraphics[width=\linewidth]{img/benchmark_availability.png}
            \caption{Availability of benchmark results for the selected models.}
            \label{fig:benchmark_availability}
        \end{figure}

\subsection{Comparing LMC Complexity and Inference Capability}

    \subsubsection{Building the testing dataset}
    \label{sec:building_testing_dataset}

        We will build a dataset by creating tuples that encode all dimensions of our analysis: model information, weight-type combination, filtering setting, the resulting LMC complexity value, the number of histogram bins used, and the available benchmark results for the model.

        The dataset construction follows a nested loop structure:

        \begin{enumerate}
            \item \textbf{For each model} in our selected set (e.g., Llama-4-Scout-17B-16E, gpt-oss-120b, etc.), we generate all non-empty subsets of the four \textbf{weight-type categories} defined in section \ref{sec:reading_model_weights} using the power set \cite{Cantor1895}. This produces 15 unique weight-type combinations per model:
            \begin{itemize}
                \item Single types: \{bias\}, \{norm\}, \{embedding\}, \{other\}
                \item Dual combinations: \{bias, norm\}, \{bias, embedding\}, \{bias, other\}, \{norm, embedding\}, \{norm, other\}, \{embedding, other\}
                \item Triple combinations: \{bias, norm, embedding\}, \{bias, norm, other\}, \{bias, embedding, other\}, \{norm, embedding, other\}
                \item All four: \{bias, norm, embedding, other\}
            \end{itemize}

            \item \textbf{For each weight-type combination}, we apply every value of \( \sigma_{\text{filter}} \) defined in section \ref{sec:filtering}. The filtering values tested are: 0.125, 0.25, 0.5, 1, 2, 3, 4, 5, 10, 20, plus one unfiltered configuration, totaling 11 configurations per combination.

            \item \textbf{For each configuration}, we compute the LMC complexity following the procedure outlined in section \ref{sec:lmc_complexity}. We also record the number of histogram bins used in the calculation for further analysis. Lastly, we append the available benchmark results for the model from section \ref{sec:benchmark_collection_procedure}.
            
        \end{enumerate}

        At the end, we will have 35 models \(\times\) 15 weight-type combinations \(\times\)  11 filtering settings (including unfiltered) = 5775 datapoints. The resulting dataset \( \mathbf{R} \) consists of tuples structured as follows:
        \begin{enumerate}
            \item Model name
            \item Number of parameters
            \item Weight-type combination
            \item Filtering setting
            \item LMC complexity value
            \item Number of histogram bins
            \item Available benchmark results
        \end{enumerate}

        e.g. a hypothetical tuple in the dataset:
        \begin{itemize}
            \item \texttt{(Llama-4-Scout-17B-16E, 17B params, \{bias\}, 0.125, 0.324, 8592, [MMLU: 71.2, MMLU-Pro: 68.5, ...])}
        \end{itemize}

    \subsubsection{Regression procedures}
    \label{sec:regression_procedures}
            
        We will also use two important concepts defined as follows:

        \begin{itemize}
            \item \textbf{Linear regression}: a conventional statistical method used to model the relationship between a dependent variable and one or more independent variables by fitting a linear equation \cite{Freedman2007}. The data may not follow a linear trend, but linear regression can still be used to identify overall correlations and make it easier to compare results.
            \begin{itemize}
                \item Linear: \( y = ax + b \)
            \end{itemize}

            For comparisons, we will use the \textbf{Pearson correlation} \cite{Freedman2007} (linear correlation coefficient). This can be computed as:
            \[ r = \frac{\text{cov}(x,y)}{SD(x) \times SD(y)} \]
            where $\text{cov}(x,y) = (\text{average of products } xy) - (\text{average of } x) \times (\text{average of } y)$.

            Or alternatively, using the linear regression slope \( a \) \cite{Wonnacott1990}:
            \[ r = a \cdot \frac{SD(x)}{SD(y)} \]

            \item \textbf{Free regression}: a curve fitting method with greater flexibility but less comparability than linear regression. It was configured to test for the following functions:
            \begin{itemize}
                \item Linear: \( y = ax + b \)
                \item Quadratic: \( y = ax^2 + bx + c \)
                \item Exponential: \( y = a \cdot e^{bx} + c \)
                \item Logarithmic: \( y = a \cdot \ln(x+1) + b x + c \)
                \item Power: \( y = a \cdot (x+1)^b + c \)
            \end{itemize}
            The best fit was chosen based on the highest \( R^2 \) value. 
            
            For both regressions, the tests were made using \texttt{curve\_fit} from SciPy~\cite{SciPy2020}.

        \end{itemize}

    \subsubsection{Influence of filtering}
    \label{sec:best_filtering_setting}

        In more detail, the criteria defined in section \ref{sec:filtering} to choose the best filtering setting work as follows:
        \begin{enumerate}
            \item We will group \( R \) in relation to each value of \( \sigma_{\text{filter}} \). Then, calculate the \textbf{average} and the \textbf{maximum} \textbf{histogram bin count} for each group, and plot them as a bar chart. 
            
            \item We will group \( R \) in relation to each value of \( \sigma_{\text{filter}} \). Then, calculate the \textbf{average} and the \textbf{maximum} \textbf{complexity} for each group, and plot them as a bar chart.

            \item Since steps 1 and 2 represent the two criteria defined in section \ref{sec:filtering}, we will first filter the groups obtained in 1 to keep only the ones that have a lower average bin count than the unfiltered data. Then, among those groups, we will choose the one with the closest complexity value to the unfiltered data (least amount of filtering).
        \end{enumerate}

        The filtering dimension will be used to determine the best general filtering setting following the criteria defined in section \ref{sec:filtering}. After defining the best filtering setting, we will discard all the runs where \( \sigma_{\text{filter}} \) is different from the chosen one, creating a new set $\mathbf{R}_{\mathbf{f}}$. There will be a total of: \( 35 \text{ models } \times 15 \text{ weight-type combinations } \times 1 \text{ filtering setting } = 525 \text{ datapoints}\).

        We will also make a free regression on a scatter plot for the data obtained in steps 1 and 2 to analyze how filtering strength affects histogram bin count and complexity, respectively.

    \subsubsection{Influence of weight types}
    \label{sec:weight_type_analysis}

        After defining the best filtering setting, we will analyze how the choice of weight types influences the final complexity results. This will help us understand if certain weight types contribute more significantly to the overall complexity measure.

        This will be done by grouping \( R_{f} \) in relation to each weight-type combination and calculating the \textbf{average} and the \textbf{maximum} \textbf{complexity} for each group, then plotting a bar chart for each metric.  
        
        It is necessary to select a best weight-type combination, like it is done in \ref{sec:best_filtering_setting}, because keeping all the different combinations will add bias to the analysis: a good way to visualize this problem is to think that a single benchmark value might be associated with 15 different complexity values for the same model, one for each weight-type combination. It also might wrongfully influence the the statistical significance by adding artificial datapoints.

        We are heavily inclined to choose the combination that uses all weight types except for the \textbf{embedding} weights, since this approach is commonly used in research such as in the scaling laws one by OpenAI \cite{Kaplan2020}. However, the final decision will be made by removing weight types with the lowest average complexity, as they are likely to contribute less to the overall complexity measure.

        After choosing the best weight-type combination, we will discard all the runs in \( R_{f} \) where the weight-type combination is different from the chosen one, creating a new set $\mathbf{R}_{\mathbf{best}}$. There will be a total of: \( 35 \text{ models } \times 1 \text{ weight-type combination } \times 1 \text{ filtering setting } = 35 \text{ datapoints}\).


    \subsubsection{Complexity vs Number of parameters}   
    \label{sec:complexity_vs_num_params}

        We will analyze how LMC complexity varies with the number of model parameters. This will be done by calculating the \textbf{average complexity} for each range of parameters and plotting it as a histogram. The number of bins of this plot will be defined using the Freedman-Diaconis rule \cite{FreedmanDiaconis1981} to adapt to the distribution.

        We will look for trends or patterns that may indicate a relationship between model size and complexity.

    \subsubsection{Complexity vs Number of bins}
    \label{sec:complexity_vs_num_bins}

        Similarly to \ref{sec:complexity_vs_num_params}, we will analyze how LMC complexity varies with the number of histogram bins used in its calculation. This will be done by calculating the \textbf{average complexity} for each range of bins and plotting it as a histogram. The number of bins of this plot will be defined using the Freedman-Diaconis rule \cite{FreedmanDiaconis1981} to adapt to the distribution.

        We will look for trends or patterns that may indicate a relationship between histogram granularity and complexity.

    \subsubsection{Complexity vs Benchmark performance}
    \label{sec:complexity_vs_benchmark_performance}

        Finally, we will analyze the relationship between LMC complexity and benchmark performance. This analysis aims to determine whether higher LMC complexity is associated with better model performance across different benchmarks, in order to validate or refute the hypothesis that complexity correlates with inference capability.
        
        For each benchmark available in \( R_{best} \), we will create scatter plots with benchmark scores on the x-axis and complexity on the y-axis.

        We will perform both linear and free regression analyses (section \ref{sec:regression_procedures}) to identify trends between complexity and performance. The strength and nature of these relationships will be assessed using Pearson correlation and \( R^2 \) values.

        As a control, the experiment will be rerun using the number of model parameters instead of LMC complexity. The number of parameters is a known variable that is expected to correlate with benchmark performance according to scaling laws \cite{Kaplan2020}. This will help validate our methodology and provide a baseline for comparison.

        We will provide a bar chart summarizing the Pearson correlation coefficient for each benchmark analyzed for both LMC complexity vs benchmark performance and number of parameters vs benchmark performance, so that they can be easily compared. In addition, for each of these plots we will check the result of the concatenation of all benchmarks available to see if a stronger correlation arises from the larger dataset, presented as another bar in the chart.

        Using the complexity's Pearson correlation results, we will check for the statistical significance of the correlations found using a two-tailed t-test \cite{Fisher1921}. The t-statistic can be computed as:

        \[ t = r \cdot \sqrt{\frac{n-2}{1-r^2}} \]

        where \( r \) is the Pearson correlation coefficient and \( n \) is the number of datapoints used in the correlation. The p-value can then be obtained from the cumulative distribution function of the $t$-distribution with $n-2$ degrees of freedom:
        
        \[ p = 2 \cdot (1 - F_t(|t|; n-2)) \]

        where \( F_t \) is the cumulative distribution function of the $t$-distribution with $n-2$ degrees of freedom. A significance level of 0.05 will be used to determine if the correlations are statistically significant. An approximation using the standard normal distribution was avoided due to the small sample size of some benchmarks (n < 30).

        As a bonus, we will also provide a top 20 list sorted by the absolute value of Pearson correlation coefficients found in the analysis, indicating which (benchmark, model, weight-type) combinations produced the strongest correlations.

        




            









\section{Results}

    \begin{figure*}[t]
        \centering
        \includegraphics[width=\textwidth]{img/average_grade_by_date_with_ipca.png}
        \caption{Average LLM sentiment grade by date and model (with IPCA inflation).}
        \label{fig:average_grade_by_date}
    \end{figure*}

    In Figure~\ref{fig:average_grade_by_date} it is possible to observe that, despite some variability, all the LLMs' sentiment follow a similar trend over time, with peaks and valleys occurring around the same dates. Even the exceptional cases such as Deepseek-chat-v3.1, which shows a significant lower average than the other models, still follows the same general trend. 
    
    This suggests that models can indeed capture market sentiment dynamics, but the bias should be taken into account when using their outputs as sentiment indicators. This also indicate that correcting for bias might improve the models' performance in downstream tasks such as inflation prediction.

    The inflation in the graph appears to be stable, but it is noticeable that some drops in sentiment happen at the same time we have peaks in inflation, such as in 2002. Interestingly, that same correlation occurs in reverse in 2022 and also don't happen at all in some other periods we would expect it to happen such as 2008.

    \begin{figure}[t]
        \centering
        \includegraphics[width=\columnwidth]{img/average_ci_by_dataset_95perc.png}
        \caption{Average grade and confidence intervals by dataset at 95\% confidence level.}
        \label{fig:average_ci_95}
    \end{figure}

    \begin{figure}[t]
        \centering
        \includegraphics[width=\columnwidth]{img/average_ci_by_dataset_99perc.png}
        \caption{Average grade and confidence intervals by dataset at 99\% confidence level.}
        \label{fig:average_ci_99}
    \end{figure}

    \begin{table}[t]
        \centering
        \begin{tabular}{lrr}
            \hline
            \textbf{Dataset} & \textbf{Average} & \textbf{Std. Dev.} \\
            \hline
            \multicolumn{3}{l}{\textit{Global}} \\
            Models only & $-0.1826$ & $0.9832$ \\
            Humans only & $-0.0413$ & $0.7187$ \\
            \hline
            \multicolumn{3}{l}{\textit{By LLM}} \\
            Claude Sonnet 4 & $-0.2572$ & $0.9664$ \\
            Deepseek Chat v3.1 & $-0.4851$ & $0.8745$ \\
            Gemini 2.5 Pro & $-0.2823$ & $0.9594$ \\
            Gemma 3 27B IT & $-0.1442$ & $0.9896$ \\
            Llama 4 Maverick & $-0.0248$ & $0.9998$ \\
            Phi 4 & $-0.1158$ & $0.9933$ \\
            GPT-5 & $-0.2146$ & $0.9768$ \\
            GPT-OSS-120B & $-0.1160$ & $0.9933$ \\
            Grok 4 Fast & $-0.0415$ & $0.9992$ \\
            \hline
            \multicolumn{3}{l}{\textit{By Human}} \\
            Specialist & $-0.0343$ & $0.6459$ \\
            Conciliated & $-0.0591$ & $0.7030$ \\
            Open & $-0.0360$ & $0.8142$ \\
            \hline
        \end{tabular}
        \caption{Average sentiment grades and standard deviations.}
        \label{tab:average_grades}
    \end{table}

    In figures \ref{fig:average_ci_95}, \ref{fig:average_ci_99} and table \ref{tab:average_grades} we can see the average grades and confidence intervals for each dataset used in the evaluation. 
    
    It is noticeable that we have a significant variation in the average grades assigned by different models, with all averages being slightly negative, including the human evaluated ones. This might indicate a general pessimistic bias in the COPOM minutes.
    
    \textbf{Grok-4-fask} and \textbf{Llama-4-maverick} were the models with a bias closer to the human averages, while \textbf{Deepseek-chat-v3.1} was the furthest and also the most pessimistic by a large margin.

    It is also noticeable that the confidence intervals are quite wide at human evaluated averages, even with lower standard deviation values, because they have a smaller number of samples. The global human evaluations are also more optimistic than the LLM ones.

    \begin{figure}[t]
        \centering
        \includegraphics[width=\columnwidth]{img/rmse_comparison_6plots.png}
        \caption{RMSE comparison across six different model configurations.}
        \label{fig:rmse_comparison}
    \end{figure}

    In figure \ref{fig:rmse_comparison} we can see the comparison of the best model configurations for each of the six setups presented.

    We can see that in most cases we have a small improvement when using sentiment grades compared to the baseline model without sentiment. While this is always true in the ARIMA setups, in the LSTM setups the results are mixed and much more unstable.

    A fascinating insight is that the most frequent best models are \textbf{Grok-4-fask} and \textbf{Llama-4-maverick}, which were also the models with sentiment averages closer to the human evaluated ones. This suggests that altering the bias towards a more human-like sentiment might improve the models' performance in inflation prediction.

    \begin{table}[h]
        \centering
        \begin{tabular}{lcc}
            \textbf{Model} & \textbf{Uncorrected} & \textbf{Corrected} \\
            \hline
            LSTM & 0.16\% & 0.23\% \\
            ARIMA & 1.20\% & 0.73\% \\
        \end{tabular}
        \caption{RMSE reduction across different models.}
        \label{tab:rmse_reduction}
    \end{table}

    In table \ref{tab:rmse_reduction}, we can observe the RMSE reduction percentages when including sentiment with and without correction compared to baseline (only inflation), an average of all our 36,792 tests.


    As we can see we had a small improvement in all configurations, with ARIMA models benefiting the most from the inclusion of sentiment overall.

    While ARIMA models observed a reduction in predition performance when using corrected sentiment grades, LSTM models saw an improvement.


    
    
    

\section{Conclusion}

It is safe to safe that the inclusion of sentiment analysis in time series forecasting models has demonstrated a measurable improvement in predictive accuracy, even if modest.

The ARIMA models, in particular, benefited significantly from the integration of sentiment data, suggesting that these models are more adept at leveraging qualitative information to enhance their forecasts. The LSTM models also showed improvement to a lesser extent.

We also observed that models whose sentiment evaluations were closer to human assessments tended to perform better in forecasting tasks. This finding highlights the importance of aligning model biases with human perspectives.

The statistical significance analysis (Table~\ref{tab:significance_results}) confirms that all observed improvements are highly significant ($p < 0.001$), providing robust evidence that the integration of sentiment analysis meaningfully enhances inflation forecasting accuracy.

\bibliographystyle{ieeetr}
\bibliography{refs}

\end{document}