\documentclass[]{preppgca}

\usepackage{lipsum}                             % para geração de dummy text

\begin{document}
\titulo{Titulo do Projeto de Pesquisa}
\aluno{nome aluno}
\emailaluno{aluno@usp.br}
\orientador{nome orientador}
\emailorientador{orientador@usp.br}
\data{\today}

%%%%%%%%%%%%%%%%%%%%%%%%%%%%%%%%%%%%%%%%%%%%%%%%%%%%%%%%%%%%%%%%%%%%%
% IMPORTANTE
%%%%%%%%%%%%%%%%%%%%%%%%%%%%%%%%%%%%%%%%%%%%%%%%%%%%%%%%%%%%%%%%%%%%%
% Se o documento for um projeto de pesquisa, ao invés de uma proposta,
% ou seja, você já estiver incrito no programa, retire o comentário
% da próxima linha.
%% \PREPPGCAtipodocumento{projeto}


\maketitle

\begin{resumo}

  \lipsum[1]
  
\end{resumo}

\section*{Introdução}
\label{intro}

\lipsum[2-3]

\section{Fundamentação Teórica}

\lipsum[4-5]

O manual do \TeX~\cite{knuth1986} pode ser usado para aprendê-lo, e o
livro do Lamport~\cite{lamport1994} para aprender o
\LaTeX\index{LaTeX}, mas se quiser ir a fundo tem que ver como o \TeX
quebra os parágrafos em linhas~\cite{knuth1981}.

\section{Objetivos}

\lipsum[6]

\section{Metodologia}

\lipsum[7]

A Figura~\ref{fig:cilindro}\index{cilindro} mostra o $\ldots$?

\begin{figure}[ht]
  \caption{Cilindro.}
  \centering
  \includegraphics{cilindro.png} 
  \label{fig:cilindro}
\end{figure}


\subsection{Plano de Trabalho}

\lipsum[8]

O plano de trabalho obedecerá a seguinte sequência:

\begin{enumerate}
\item Cumprimento dos créditos e estudo do referencial teórico;
\item Elaboração do modelo;
\item Desenvolvimento, teste e análise;
\item Escrita e defesa da dissertação.
\end{enumerate}

A Tabela~\ref{tab:crono} apresenta o cronograma de execução para o
plano de trabalho.

\def\mark{$\star$}
\begin{table}[htbp]
\centering
\begin{tabular}{|l|l|l|l|l|l|l|l|}
\hline
                         & \multicolumn{4}{c|}{\textbf{Atividade}} \\ \hline
\textbf{Semestre}        & 1     & 2     & 3     &    4    \\ \hline
1/2021 & \mark &       &       &         \\ \hline
2/2021 & \mark & \mark & \mark &         \\ \hline
1/2022 &       &       & \mark &         \\ \hline
2/2022 &       &       &       &  \mark  \\ \hline
\end{tabular}
\caption{Cronograma de execução do plano de trabalho.}
\label{tab:crono}
\end{table}

\bibliography{refs}

\end{document}
