\section{Materials and Methods}

    \subsection{Model Complexity}

        The complexity measure used in this study is based on the López-Ruiz-Mancini-Calbet (LMC) statistical complexity measure~\cite{lopez1995statistical, martin2003statistical}, following recent research that demonstrates the relationship between complexity and inference performance~\cite{murta2025complexity}. For a neural network with weight parameters $\theta = \{\theta_1, \theta_2, \ldots, \theta_n\}$, we define the complexity measure as follows:

        \textbf{Statistical Complexity of Model Weights:}
        The LMC complexity $C_{LMC}(\theta)$ of a model's weight configuration is defined as:

        \begin{equation}
            C_{LMC}(\theta) = H(\theta) \times D(\theta)
        \end{equation}

        where $H(\theta)$ is the normalized Shannon entropy and $D(\theta)$ is the disequilibrium measure. The normalized Shannon entropy is computed as:

        \begin{equation}
            H(\theta) = -\frac{1}{\log_2 n} \sum_{i=1}^{n} p_i \log_2 p_i
        \end{equation}

        and the disequilibrium is defined as:

        \begin{equation}
            D(\theta) = \sum_{i=1}^{n} \left(p_i - \frac{1}{n}\right)^2
        \end{equation}

        where $p_i$ represents the normalized probability distribution of weight bins obtained by discretizing the weight space, and $n$ is the number of bins. This measure captures both the information content (entropy) and the degree of structure (disequilibrium) in the weight distribution.

        The computation will be implemented using tools provided by frameworks like PyTorch~\cite{paszke2019pytorch}. The chosen discretization approach and bin selection strategy will be explained in the final presentation.

    \subsection{Inference Capability}

        Inference capability will be assessed using standardized benchmarks and evaluation metrics commonly used in the machine learning community~\cite{liang2022holistic, wang2019superglue}. This requires careful investigation of comparison methodologies since benchmarks often become obsolete quickly due to rapid model improvements~\cite{raji2021ai}.

    \subsection{Methodology}

        The research will follow the following step-by-step approach:

        \textbf{Step 1 - Model Selection}
        \begin{itemize}
            \item Select publicly available open-source models with accessible weights. Models available through platforms like HuggingFace
        \end{itemize}

        \textbf{Step 2 - Literature Review and Benchmark Data Collection}
        \begin{itemize}
            \item Search for existing benchmark results for the selected models
            \item Conduct literature review on model evaluation methodologies
            \item Determine the most appropriate inference measures based on data availability
        \end{itemize}

        \textbf{Step 3 - Complexity Measure Extraction}
        \begin{itemize}
            \item Utilize tools such as PyTorch~\cite{paszke2019pytorch} to extract and calculate the model complexity measure from model weights
        \end{itemize}

        \textbf{Step 4 - Statistical Analysis and Relationship Modeling}
        \begin{itemize}
            \item Structure results of the relationship between complexity and model performance
            \item Include measures of statistical significance (correlation analysis, regression modeling, confidence intervals...)
        \end{itemize}
