\chapter{Considerações Finais}

O desenvolvimento desta aplicação para análise automatizada do crescimento de plantas demonstra o potencial do processamento de imagens aliado a uma interface gráfica intuitiva para modernizar práticas de monitoramento vegetal. O sistema mostrou-se eficaz na segmentação de folhas e na extração de métricas morfológicas relevantes, permitindo o acompanhamento quantitativo do desenvolvimento das plantas ao longo do tempo.

A flexibilidade e acessibilidade da solução tornam-na adequada tanto para ambientes de pesquisa quanto para uso didático ou por produtores. A organização em coleções, o processamento em lote e a geração automática de gráficos facilitam o gerenciamento de experimentos e a visualização dos resultados.

Apesar dos avanços, algumas limitações persistem, especialmente na detecção do caule, que se mostrou um desafio técnico relevante. Futuras melhorias podem envolver o uso de técnicas mais avançadas de aprendizado de máquina, integração com sensores adicionais ou calibração para conversão de pixels em medidas reais.

Em síntese, a aplicação representa um passo importante para a automação e padronização da análise de crescimento vegetal, contribuindo para decisões mais informadas e para o avanço da Agricultura Inteligente.
