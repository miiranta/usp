\chapter{Introdução}

A Agricultura Inteligente é a aplicação de tecnologias avançadas ao setor agrícola, com o objetivo de tornar as práticas de cultivo mais eficientes e sustentáveis. Alguns casos práticos de aplicação desses sistemas são sistemas de gerenciamento de irrigação, uso controlado de fertilizantes, e monitoramento da qualidade da plantação. O uso de câmeras na agricultura inteligente é cada vez mais comum e, com isso, inúmeras novas possibilidades são permitidas no controle e acompanhamento da produção.

A obtenção de dados é um dos cernes da agricultura inteligente. Como tal, precisamos de meios de qualificar e quantificar as melhorias obtidas com uso de ferramentas novas, técnicas mais avançadas e recursos diferentes. Visando criar e aprimorar métodos existentes de análise das plantações, os autores propõem um método eficiente de base para, a partir de imagens relativamente complexas para processamento, obter a área, a altura e a largura das plantas representadas, de modo que seja possível acompanhar e avaliar o crescimento destas. 

As aplicações deste método, porém, vão consideravelmente além deste escopo: criação de benchmarks para avaliar cientificamente o desenvolvimento, quantificar a estimativa de retorno de safras agrícolas, monitorar a saúde das culturas em tempo real, detectar precocemente sinais de pragas ou doenças, otimizar o uso de recursos como água e fertilizantes, planejar a logística de colheitas de forma mais eficiente, e apoiar decisões estratégicas de manejo agrícola com base em dados concretos.
