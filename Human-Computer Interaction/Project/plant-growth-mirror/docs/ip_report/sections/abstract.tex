\chapter*{Resumo}
 
No cenário agrícola atual de pressão por maior produtividade e uso racional de recursos, a Agricultura Inteligente surge como ponte entre tecnologia e sustentabilidade. Neste trabalho, propomos um método automatizado de extração de métricas de crescimento de plantas (área, altura e largura) a partir de imagens digitais. A metodologia combina segmentação com análise geométrica. O sistema demonstra capacidade de gerar indicadores confiáveis para monitoramento contínuo em diferentes tipos de imagens, permitindo detecção de estresses bióticos e abióticos, planejamento de irrigação e insumos, além de servir de base para benchmarks de produtividade. Desenvolvida como uma interface intuitiva, a solução facilita o acesso de produtores e pesquisadores a dados objetivos, promovendo decisões mais informadas na agricultura.
