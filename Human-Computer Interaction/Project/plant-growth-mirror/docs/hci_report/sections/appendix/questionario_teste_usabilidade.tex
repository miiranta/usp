\chapter{Roteiro de Teste de Usabilidade}
\label{chap:roteiro_teste_usabilidade}

\section*{1. Sobre o Sistema}

Este sistema foi desenvolvido para auxiliar no acompanhamento e análise de plantas ao longo do tempo, a partir do envio de imagens. Por meio do processamento dessas imagens, o sistema gera informações visuais e numéricas, permitindo que o usuário visualize a evolução das plantas, organize imagens em coleções e personalize parâmetros de análise de acordo com suas necessidades.

\section*{2. Tarefas}

Para realizar o Teste de Usabilidade, abaixo estão descritas as principais tarefas e cenários de uso do sistema:

\begin{itemize}
    \item \textbf{Adicionar imagem ao sistema}: Enviar uma ou mais imagens de plantas para o sistema a partir do computador.
    
    \item \textbf{Processar uma imagem}: Aplicar o processamento da imagem para extrair informações como área, altura e largura da planta.
    
    \item \textbf{Remover uma imagem do sistema}: Excluir permanentemente uma imagem previamente adicionada.
    
    \item \textbf{Editar uma imagem}: Alterar a data associada à imagem ou modificar os parâmetros de processamento aplicados.
    
    \item \textbf{Criar uma coleção}: Criar um agrupamento de imagens que representem uma mesma planta ou experimento ao longo do tempo.
    
    \item \textbf{Renomear uma coleção}: Modificar o nome de uma coleção já existente.
    
    \item \textbf{Associar uma imagem a uma coleção}: Incluir uma imagem em uma coleção específica para organizá-la junto a outras relacionadas.
    
    \item \textbf{Desassociar uma imagem de uma coleção}: Remover a ligação entre uma imagem e uma determinada coleção, sem apagar a imagem.
    
    \item \textbf{Excluir uma coleção}: Apagar uma coleção do sistema, o que pode ou não remover as imagens associadas a ela (dependendo da implementação).
\end{itemize}

\section*{3. Questionário Final}

\textbf{Instruções:} Marque a alternativa que melhor representa sua opinião, utilizando a escala de 1 (discordo totalmente) a 5 (concordo totalmente).

\vspace{1em}

\begin{tabularx}{\textwidth}{>{\raggedright\arraybackslash}X c c c c c c}
\textbf{Afirmação} & \textbf{1} & \textbf{2} & \textbf{3} & \textbf{4} & \textbf{5} \\
\hline
1. Eu achei o sistema fácil de usar & $\Box$ & $\Box$ & $\Box$ & $\Box$ & $\Box$ \\
2. As funcionalidades estavam organizadas de forma lógica & $\Box$ & $\Box$ & $\Box$ & $\Box$ & $\Box$ \\
3. Eu me senti confiante ao usar o sistema & $\Box$ & $\Box$ & $\Box$ & $\Box$ & $\Box$ \\
4. A linguagem utilizada no sistema era clara e compreensível & $\Box$ & $\Box$ & $\Box$ & $\Box$ & $\Box$ \\
5. A aparência do sistema era agradável & $\Box$ & $\Box$ & $\Box$ & $\Box$ & $\Box$ \\
6. O sistema me deu feedback suficiente sobre minhas ações & $\Box$ & $\Box$ & $\Box$ & $\Box$ & $\Box$ \\
7. Tive dificuldade para concluir as tarefas que me foram solicitadas & $\Box$ & $\Box$ & $\Box$ & $\Box$ & $\Box$ \\
8. Eu entendi os resultados gerados pelo sistema após o processamento & $\Box$ & $\Box$ & $\Box$ & $\Box$ & $\Box$ \\
\end{tabularx}

\vspace{1em}

\noindent\textbf{O que você mais gostou no sistema?} \\
Resposta: \rule{14cm}{0.4pt}

\vspace{1em}

\noindent\textbf{O que você menos gostou ou teve mais dificuldade em fazer?} \\
Resposta: \rule{14cm}{0.4pt}

\vspace{1em}

\noindent\textbf{Há algo que você mudaria na interface ou na forma como as informações são apresentadas?} \\
Resposta: \rule{14cm}{0.4pt}

\vspace{1em}

\noindent\textbf{Algum comentário adicional?} \\
Resposta: \rule{14cm}{0.4pt}
