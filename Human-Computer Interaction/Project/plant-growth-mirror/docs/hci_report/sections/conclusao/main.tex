\chapter{Conclusão}

O desenvolvimento do Plant Growth Analyzer proporcionou uma experiência abrangente de aplicação dos princípios de Interação Humano-Computador (IHC) em um contexto real de análise de crescimento vegetal. O projeto foi guiado por diretrizes de usabilidade, princípios de design e leis de UX, resultando em uma ferramenta funcional, acessível e tecnicamente robusta.

A análise dos requisitos e casos de uso demonstrou que o sistema atende às necessidades centrais de pesquisadores e entusiastas, permitindo o gerenciamento eficiente de imagens, processamento automatizado e visualização clara dos dados de crescimento. A arquitetura modular e a interface baseada em componentes garantiram flexibilidade, manutenção facilitada e consistência visual.

As inspeções de usabilidade e os testes com usuários revelaram pontos fortes, como a performance, a organização visual e a adequação das funcionalidades principais. No entanto, também evidenciaram oportunidades de melhoria, especialmente em relação à consistência visual, feedback do sistema, acessibilidade e clareza de navegação. As recomendações levantadas — como padronização de cores, aprimoramento de feedback e inclusão de documentação contextual — são fundamentais para elevar ainda mais a experiência do usuário.

Os testes automatizados com Lighthouse confirmaram a excelência em desempenho e boas práticas técnicas, mas reforçaram a necessidade de ajustes em contraste de cores e estabilidade de layout dinâmico.

Em síntese, o Plant Growth Analyzer cumpre seu objetivo de democratizar a análise de crescimento de plantas, oferecendo uma solução acessível e eficiente. A continuidade do aprimoramento, guiada pelas avaliações realizadas, garantirá que a aplicação se torne cada vez mais intuitiva, inclusiva e alinhada às melhores práticas de IHC.
