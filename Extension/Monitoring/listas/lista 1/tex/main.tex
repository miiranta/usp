\documentclass[12pt,a4paper]{article}
\usepackage[utf8]{inputenc}
\usepackage[T1]{fontenc}
\usepackage[brazil]{babel}
\usepackage{amsmath,amssymb}
\usepackage{enumitem}
\usepackage{hyperref}
\usepackage{geometry}
\usepackage{parskip}

\geometry{top=25mm,left=25mm,right=25mm,bottom=25mm}

\title{Lista de Exercícios I - Monitoria}
\author{Arquitetura de Computadores}
\date{}

\begin{document}
\maketitle
\noindent Para realizar os exercícios a seguir, instale o simulador \href{https://github.com/Simulador-Simus/SimuS}{SimuS}. Consulte também a \href{https://github.com/Simulador-SimuS/SimuS/blob/master/docs/simus.pdf}{referência da arquitetura utilizada}.

\begin{enumerate}[left=0pt,label=\textbf{Exercício \arabic*:},itemsep=8pt]

\item Implemente um gerador da sequência de Fibonacci sem utilizar instruções de pilha (POP/ PUSH).

Requisitos mínimos:
\begin{itemize}
	\item inicializar os dois primeiros termos da sequência (F(0) e F(1));
	\item calcular termos subsequentes iterativamente sem usar POP/ PUSH;
	\item armazenar o termo atual em uma variável acessível em memória.
\end{itemize}

Saída esperada: o programa deve produzir os termos da sequência de Fibonacci em ordem crescente na memória, um termo por iteração; não é necessário implementar limite superior; o programa deve progredir iterativamente.

\item Implemente um programa que some dois valores inteiros armazenados em memória.

A variável \texttt{NUM1} contém o primeiro operando e \texttt{NUM2} contém o segundo operando. Some os dois valores e armazene o resultado em \texttt{RESULTADO}.

Requisitos mínimos:
\begin{itemize}
	\item ler \texttt{NUM1} e \texttt{NUM2};
	\item executar a operação de soma;
	\item armazenar o resultado em \texttt{RESULTADO};
\end{itemize}

Saída esperada: após execução, a variável \texttt{RESULTADO} conterá a soma de \texttt{NUM1} e \texttt{NUM2}.

\item Implemente um programa que verifique se um valor inteiro armazenado em memória é par ou ímpar.

O número de entrada está na variável \texttt{NUMERO}. Se \texttt{NUMERO} for par, armazene 1 em \texttt{RESULTADO}; caso contrário, armazene 0.

Requisitos mínimos:
\begin{itemize}
	\item ler um número;
	\item utilizar operações lógicas (por exemplo \texttt{AND} com 1) ou aritméticas para determinar paridade;
	\item usar desvios condicionais (\texttt{JZ}/\texttt{JN}) para controlar o fluxo;
	\item escrever o resultado em \texttt{RESULTADO} (1 = par, 0 = ímpar).
\end{itemize}

Saída esperada: após execução, \texttt{RESULTADO} conterá 1 se \texttt{NUMERO} for par ou 0 se for ímpar.

\item Implemente um programa que conte de 1 até 5 utilizando um loop e armazene o valor atual em memória.

Requisitos mínimos:
\begin{itemize}
	\item inicializar a variável \texttt{CONTADOR} com 1;
	\item incrementar \texttt{CONTADOR} a cada iteração utilizando instruções aritméticas;
	\item empregar comparações e desvios condicionais (por exemplo \texttt{JN}/\texttt{JZ}) para controlar o término do loop.
\end{itemize}

Saída esperada: ao final da execução, a variável \texttt{CONTADOR} deve conter o valor 5.

\item Implemente um programa que utilize endereçamento indireto para copiar um valor de memória.

A variável \texttt{PONTEIRO} contém o endereço de memória onde está o valor de origem. Use o modo de endereçamento indireto para carregar o valor apontado por \texttt{PONTEIRO} e copie-o para a variável \texttt{DESTINO}.

Requisitos mínimos:
\begin{itemize}
	\item definir um valor origem em memória;
	\item definir \texttt{PONTEIRO} apontando para esse endereço;
	\item usar instrução com endereçamento indireto para ler o valor e em seguida armazená-lo em \texttt{DESTINO}.
\end{itemize}

Saída esperada: após execução, a variável \texttt{DESTINO} deve conter a cópia do valor de origem.

\item Implemente um programa que aplique uma máscara de bits por meio de operação lógica \texttt{AND}.

O operando está na variável \texttt{NUMERO} e a máscara está em \texttt{MASCARA}. Armazene o resultado da operação em \texttt{RESULTADO}.

Requisitos mínimos:
\begin{itemize}
	\item ler \texttt{NUMERO} e \texttt{MASCARA} do espaço de dados;
	\item aplicar a operação lógica bit a bit \texttt{AND} entre \texttt{NUMERO} e \texttt{MASCARA};
	\item armazenar o resultado em \texttt{RESULTADO}.
\end{itemize}

Saída esperada: após execução, \texttt{RESULTADO} conterá o valor de \texttt{NUMERO} \texttt{AND} \texttt{MASCARA}.

\item Implemente um programa que utilize deslocamento lógico à esquerda para multiplicar um valor por 2.

O valor de entrada está na variável \texttt{NUMERO}. Aplique a instrução \texttt{SHL} e armazene o resultado em \texttt{RESULTADO}.

Requisitos mínimos:
\begin{itemize}
	\item ler \texttt{NUMERO} do espaço de dados;
	\item aplicar instrução de deslocamento lógico à esquerda (\texttt{SHL}) para multiplicar por 2;
	\item armazenar o resultado em \texttt{RESULTADO};
\end{itemize}

Saída esperada: após execução, \texttt{RESULTADO} conterá \texttt{NUMERO} multiplicado por 2. Desconsidere overflow.

\item Implemente um programa que realize subtração com borrow utilizando a instrução \texttt{SBC}.

Calcule \texttt{NUM1} - \texttt{NUM2} considerando o estado inicial do carry/borrow conforme a arquitetura; defina o carry/borrow inicial antes de executar \texttt{SBC}.

Requisitos mínimos:
\begin{itemize}
	\item carregar \texttt{NUM1} e \texttt{NUM2};
	\item definir o carry/borrow inicial conforme necessário;
	\item executar a instrução \texttt{SBC} para efetuar a subtração com borrow;
	\item armazenar o resultado em \texttt{RESULTADO}.
\end{itemize}

Saída esperada: após execução, \texttt{RESULTADO} conterá \texttt{NUM1} - \texttt{NUM2} (com borrow aplicado).

\item Implemente um programa que utilize a operação lógica \texttt{XOR} para alternar bits de um valor.

O operando está na variável \texttt{NUMERO} e a máscara está em \texttt{MASCARA\_XOR}. Aplique \texttt{NUMERO} \texttt{XOR} \texttt{MASCARA\_XOR} e armazene o resultado em \texttt{RESULTADO}.

Requisitos mínimos:
\begin{itemize}
	\item ler \texttt{NUMERO};
	\item aplicar a operação \texttt{XOR} entre \texttt{NUMERO} e \texttt{MASCARA\_XOR};
	\item armazenar o resultado em \texttt{RESULTADO}.
\end{itemize}

Saída esperada: após execução, \texttt{RESULTADO} conterá \texttt{NUMERO} \texttt{XOR} \texttt{MASCARA\_XOR}.

\item Implemente um programa que demonstre o uso da pilha por meio das instruções \texttt{PUSH} e \texttt{POP}.

Empilhe o valor armazenado em \texttt{VALOR\_A}, carregue \texttt{VALOR\_B} no acumulador, em seguida desempilhe o valor de \texttt{VALOR\_A} e some-o ao acumulador.

Requisitos mínimos:
\begin{itemize}
	\item utilizar instruções \texttt{PUSH} e \texttt{POP} para preservar e restaurar valores na pilha;
	\item carregar \texttt{VALOR\_B} no acumulador antes de desempilhar \texttt{VALOR\_A};
	\item efetuar a soma e armazenar o resultado em \texttt{RESULTADO}.
\end{itemize}

Saída esperada: após execução, \texttt{RESULTADO} conterá \texttt{VALOR\_A} + \texttt{VALOR\_B}.
\end{enumerate}

\end{document}

