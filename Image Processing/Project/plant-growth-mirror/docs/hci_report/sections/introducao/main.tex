\chapter{Introdução}

O \textbf{Plant Growth Analyzer} é uma aplicação desktop desenvolvida em Angular e Electron que permite aos usuários analisar o crescimento de plantas através de fotografias semanais. O sistema oferece uma solução para pesquisadores, agricultores e entusiastas que desejam monitorar e quantificar o desenvolvimento vegetal em altura, largura e área.

A aplicação combina tecnologias de desenvolvimento web com processamento de imagem, utilizando algoritmos de segmentação e análise de cor para extrair as métricas das plantas fotografadas. O sistema é projetado para ser intuitivo e eficiente, permitindo que usuários com diferentes níveis de expertise técnica possam realizar as análises.

\section{Objetivo}

O objetivo principal do Plant Growth Analyzer é fornecer uma ferramenta acessível para:

\begin{enumerate}
    \item \textbf{Monitoramento}: Permitir o acompanhamento regular do crescimento de plantas através de fotografias padronizadas
    \item \textbf{Análise Quantitativa}: Extrair métricas objetivas como altura, largura e área das plantas analisadas
    \item \textbf{Organização de Dados}: Facilitar o gerenciamento de múltiplas coleções de fotos e suas respectivas análises
    \item \textbf{Visualização de Tendências}: Gerar gráficos evolutivos que permitam identificar padrões de crescimento ao longo do tempo
    \item \textbf{Processamento Automatizado}: Reduzir o trabalho manual do nosso público alvo
\end{enumerate}

O sistema visa democratizar o acesso a técnicas de análise de crescimento vegetal, tradicionalmente restritas e caras, tornando-as acessíveis a um público mais amplo.

\section{Características e Funcionalidades}

O programa oferece um conjunto de funcionalidades que podem ser organizadas em algumas categorias principais:

\subsection{Gerenciamento de Imagens}

\begin{itemize}
    \item \textbf{Upload Individual e Múltiplo}: Suporte para envio de uma ou múltiplas imagens simultaneamente
    \item \textbf{Processamento Automático}: Análise automática das imagens com extração de métricas (altura, largura, área)
    \item \textbf{Configurações Personalizáveis}: Ajuste de parâmetros como granularidade de segmentação e threshold de verde
    \item \textbf{Metadados Inteligentes}: Extração automática de datas dos arquivos de imagem ou uso da data atual
    \item \textbf{Visualização de Resultados}: Exibição lado a lado da imagem original e processada com métricas extraídas
    \item \textbf{Edição Pós-processamento}: Possibilidade de ajustar configurações e datas após o processamento inicial
\end{itemize}

\subsection{Gerenciamento de Coleções}

\begin{itemize}
    \item \textbf{Criação de Coleções}: Organização de imagens em grupos temáticos ou experimentais
    \item \textbf{Visualização de Coleções}: Interface dedicada para explorar o conteúdo de cada coleção
    \item \textbf{Associação de Imagens}: Adição e remoção de fotos de coleções existentes
    \item \textbf{Edição de Metadados}: Renomeação de coleções
    \item \textbf{Exclusão Segura}: Remoção de coleções com confirmação de segurança
    \item \textbf{Gráficos Evolutivos}: Visualização temporal das métricas de crescimento
    \item \textbf{Comparação Temporal}: Análise de tendências de crescimento ao longo do tempo
\end{itemize}

\subsection{Características Técnicas}

\begin{itemize}
    \item \textbf{Arquitetura Desktop}: Aplicação nativa para Windows, macOS e Linux via Electron
    \item \textbf{Processamento Local}: Análise realizada localmente sem dependência de serviços externos
    \item \textbf{Algoritmos Avançados}: Utilização de técnicas de segmentação de imagem e análise de cor
    \item \textbf{Interface Moderna}: Design baseado em componentes reutilizáveis, seguindo regras consagradas de design
    \item \textbf{Performance Otimizada}: Carregamento rápido e processamento eficiente de imagens
\end{itemize}

O sistema foi desenvolvido com foco na usabilidade, combinando funcionalidades de processamento de imagem com uma interface intuitiva que permite aos usuários concentrarem-se na análise dos dados ao invés de lidar com complexidades técnicas. 